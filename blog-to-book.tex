%!TEX TS-program = xelatex
%!TEX encoding = UTF-8 Unicode

\documentclass[12pt]{book}
\usepackage{b2b/blog-to-book}

\title{สร้างหนังสือ GotoKnow Blog-to-Book ด้วย \ltc{\LaTeX}}
\author{
\textbf{ผศ.ดร.ธวัชชัย ปิยะวัฒน์} \\
\texttt{GotoKnow.org/profile/thawatchai}
}

\date{}

\usepackage{fancyhdr}
\pagestyle{fancy}
\rhead{}
\lhead{}
\renewcommand\headrulewidth{0.0pt}
%\lfoot{ผศ.ดร.ธวัชชัย ปิยะวัฒน์}
%\cfoot{\thepage}
%\rfoot{สำนักพิมพ์มหาวิทยาลัยสงขลานครินทร์}

\begin{document}

\maketitle

\setcounter{page}{1}
\pagenumbering{roman}
\tableofcontents
\newpage
\setcounter{page}{1}
\pagenumbering{arabic}


\posttitle{คำนำ}

โครงการ Blog-to-Book ของ GotoKnow.org เป็นโครงการที่ดีอย่างยิ่ง แต่ค่าจัดรูปเล่มที่ดีไซน์เนอร์เรียกมาสำหรับแต่ละเล่มนั้นเป็นราคาที่ GotoKnow.org ไม่สามารถจ่ายได้ในระยะยาว ด้วยเหตุนี้ เราจึงทดลองอะไรบางอย่างที่แปลกประหลาด นั่นคือการเอา \ltc{\LaTeX} มาใช้ในการจัดเอกสารสำหรับโครงการนี้ ปรากฎว่า \ltc{\LaTeX} ใช้ได้และใช้ได้ดีมาก ขอบคุณซอฟต์แวร์ดี ๆ อย่าง \ltc{\LaTeX}

\posttitle{โครงสร้างหนังสือ blog-to-book}

หนังสือ blog-to-book เล่มหนึ่ง ๆ จะประกอบด้วยเนื้อหาอย่างน้อยดังหัวข้อต่อไปนี้

\begin{description}

\item[ปก] ดูตามตัวอย่าง ระบบจะสร้างให้อัตโนมัติ

\item[สารบัญ] ดูตามตัวอย่าง ระบบจะสร้างให้อัตโนมัติ

\item[คำนำ] โดยผู้เขียน

\item[เกี่ยวกับผู้เขียน] เขียนโดยผู้เขียน

\item[บันทึก] เรียงตามลำดับตามที่ผู้เขียนส่งต้นฉบับมา

\end{description}

\posttitle{ซอฟต์แวร์ที่ต้องใช้}

ซอฟต์แวร์ที่ต้องใช้ในการจัดเอกสารด้วย \ltc{\LaTeX} กับเครื่องคอมพิวเตอร์ที่ใช้ระบบปฎิบัติการ Microsoft\textregistered\ Windows\texttrademark\ มีดังนี้

\begin{description}

\item[MikTeX] \texttt{http://miktex.org/}

\item[DIP SIPA Fonts] ติดตั้งได้จาก \texttt{http://www.thaiopensource.org/download/}ฟอนต์\texttt{-sipa-dip}

\item[Git for Windows] เพื่อ download ไฟล์ช่วยสร้างหนังสือ

\item[blog-to-book] download ด้วย git จาก \url{http://github.com/thawatchai/blog-to-book} ควรใช้ git อย่าใช้ ``Download Source'' ที่มุมขวาบน เพราะจะทำให้การ upgrade ลำบาก

\end{description}

\posttitle{โครงสร้าง folder ของหนังสือ}

ใน folder ของหนังสือเล่มหนึ่ง ๆ จะประกอบด้วยไฟล์ต่อไปนี้

\begin{itemize}
\item \texttt{blog-to-book.tex}
\item \texttt{blog-to-book.pdf}
\end{itemize}

ไฟล์สองไฟล์นี้เป็นไฟล์ของเอกสารนี้

และไฟล์ใน folder ย่อย b2b/

\begin{itemize}
\item \texttt{b2b/blog-to-book.sty}
\item \texttt{b2b/blog-to-book-begin.tex}
\item \texttt{b2b/blog-to-book-end.tex}
\end{itemize}

ไฟล์เหล่านี้คือไฟล์เพื่อช่วยในการทำหนังสือ

ผู้จัดหนังสือจะสร้างไฟล์ของหนังสือนั้น ๆ เช่น \texttt{abc.tex}ใน folder หลักและพิมพ์แก้ไขเพิ่มเติมข้อมูลของหนังสือในไฟล์นี้ หลังจากสั่งโปรแกรมทำงานเพื่อสร้างไฟล์หนังสือจะได้ไฟล์ PDF และไฟล์ประกอบอื่น ๆ ไฟล์เหล่านั้นสามารถลบทิ้งได้

\posttitle{tags สำหรับ blog-to-book}

ในการจัดบันทึกหนึ่ง ๆ สำหรับหนังสือ blog-to-book นั้น ผู้จัดต้องใช้ tags ต่าง ๆ ต่อไปนี้

\begin{description}

\item[\textbackslash posttitle] tag นี้มีไว้ใส่ชื่อบันทึก ใช้ดังนี้ \texttt{\textbackslash posttitle\{title text\}} (ให้ใช้ postinfo แทน)

\item[\textbackslash postdate] tag นี้มีไว้เพื่อใส่เวลาเขียนบันทึกนั้น ใช้ดังนี้ \texttt{\textbackslash postdate\{date\}} (ให้ใช้ postinfo แทน)

\item[\textbackslash postauthor] tag นี้มีไว้เพื่อใส่ชื่อผู้เขียนนั้น ใช้ดังนี้ \texttt{\textbackslash postauthor\{author\}} (ให้ใช้ postinfo แทน)

\item[\textbackslash posturl] tag นี้มีไว้เพื่อใส่ URL ของบันทึกนั้น ใช้ดังนี้ \texttt{\textbackslash posturl\{url\}} (ให้ใช้ postinfo แทน)

\item[\textbackslash postinfo] เป็น tag รวมของ tag ทั้งหมดข้างต้น ใช้ดังนี้ \texttt{\textbackslash postinfo\{title\}\{author\}\{url\}\{date\}}

\item[\textbackslash link] tag นี้มีไว้เพื่อใส่ลิงก์ในเนื้อหาของบันทึก ใช้ดังนี้ \texttt{\textbackslash link\{Giving = Happiness\}\{http://www.worldchanging.com/archives/007907.html\}} เมื่อใส่แล้วลิงก์จะกลายเป็น footnote ของเนื้อหานั้น (\link{ดูตัวอย่าง}{http://www.worldchanging.com/archives/007907.html})

\end{description}

นอกจากนี้ผู้จัดสามารถใช้ tag อื่น ๆ ของ \ltc{\LaTeX} ได้ตามปกติ อาทิเช่น

\begin{description}

\item[\textbackslash textbf] tag นี้มีไว้เน้นความสำคัญทำให้เป็น \textbf{ตัวหนา} ใช้ดังนี้ \texttt{\textbackslash textbf\{text to bold\}}

\item[\textbackslash emph] tag นี้มีไว้เน้นความสำคัญทำให้เป็น \emph{ตัวเอียง} ใช้ดังนี้ \texttt{\textbackslash emph\{text to italic\}}

\item[includegraphics] ดูรายละเอียดจาก \url{http://en.wikibooks.org/wiki/LaTeX/Importing_Graphics}

\item[description]

\item[itemize]

\item[enumerate]

\item[table]

\end{description}

อ่าน \url{http://en.wikibooks.org/wiki/LaTeX/Formatting} เพื่อรายละเอียดเพิ่มเติม

\posttitle{โครงสร้างไฟล์หนังสือ}

หนังสือหนึ่งเล่มจะเริ่มต้นด้วยโครงสร้างไฟล์ดังต่อไปนี้

\begin{verbatim}

%!TEX TS-program = xelatex
%!TEX encoding = UTF-8 Unicode

\documentclass[12pt]{book}
\usepackage{b2b/blog-to-book}

\title{book title goes here}
\author{
\textbf{author name is here} \\
\texttt{GotoKnow.org/profile/<author's username>}
}

\date{}

\usepackage{fancyhdr}
\pagestyle{fancy}
\rhead{}
\lhead{}
\renewcommand\headrulewidth{0.0pt}
%\lfoot{ผศ.ดร.ธวัชชัย ปิยะวัฒน์}
%\cfoot{\thepage}
%\rfoot{สำนักพิมพ์มหาวิทยาลัยสงขลานครินทร์}

\begin{document}

\maketitle

\setcounter{page}{1}
\pagenumbering{roman}
\tableofcontents
\newpage
\setcounter{page}{1}
\pagenumbering{arabic}


% book contents go here ....................

\end{document}

\end{verbatim}

ในแต่ละบันทึก ต้องจัดโดยมีโครงสร้างดังต่อไปนี้

\begin{verbatim}

\posttitle{post title here}

\postdate{post date here}

paragraph 1 ................

paragraph 2 .................................

paragraph 3 .....................................................

\end{verbatim}

โดยในเนื้อหาของแต่ละ paragraph นั้นสามารถใช้ link, textbf, emph, itemize, description, และ \ltc{\LaTeX} tag อื่น ๆ ได้ตามที่ผู้เขียนบันทึกนั้น ๆ ทำไว้ในเว็บไซต์

ส่วน ``คำนำ" และ ``เกี่ยวกับผู้เขียน" ให้จัดเสมือนว่าเป็นบันทึกหนึ่ง

\posttitle{เรื่องต้องระวัง}

\begin{itemize}

\item การใส่เครื่องหมายคำพูดนั้น เครื่องหมายเปิดต้องใส่ด้วย \texttt{``} และเครื่องหมายปิดด้วย \texttt{"} จะได้ผลลัพธ์เป็น ``ดังนี้'' หากใส่ \texttt{"} ทั้งหัวท้ายแล้วจะได้ผลลัพธ์เป็น "ดังนี้"

\item ประโยคยาว ๆ ที่ไม่ได้เว้นวรรคเลย ระบบจะตัดคำไม่ถูกต้อง หรือไม่ก็จะจัดกั้นหลังได้ไม่สวย ให้เพิ่มวรรค หรือปรับประโยคให้เว้นวรรคได้ (ควรถามผู้เขียนก่อนด้วย)

\end{itemize}

\postinfo{ตัวอย่าง: พิสูจน์แล้ว การให้มีความสุขกว่าการรับ}{ผศ.ดร.ธวัชชัย ปิยะวัฒน์}{http://gotoknow.org/blog/averageline/175334}{4 เมษายน 2551}

ข่าวใหญ่ประจำวันนี้ครับ เป็นข่าวใหญ่ในความรู้สึกผม แต่ไม่ได้เป็นข่าวใหญ่ที่สำนักข่าวไหน ๆ นำไปเป็นข่าวสำคัญครับ

ข่าวนั้นคือ ``นักวิทยาศาสตร์พิสูจน์แล้ว การให้มีความสุขกว่าการรับ เพราะการให้คือความสุขที่แท้จริง"

ผมอ่านข่าวนี้มาจาก \link{Giving = Happiness}{http://www.worldchanging.com/archives/007907.html} เป็นข่าวที่น่ายินดีเป็นอย่างยิ่ง

นักวิจัยจาก University of British Columbia (UBC) ได้ค้นพบว่า ``การให้มีความสุขกว่าการรับ" ครับ โดยใช้การทดลองกับกลุ่มตัวอย่างโดยให้เงินแล้วทดสอบทางจิตวิทยาในภายหลัง พบว่ากลุ่มที่ให้นำเงินไปใช้ประโยชน์แก่คนอื่นเป็นกลุ่มที่สมองส่วนแสดงความสุขทำงานดีกว่ากลุ่มที่ให้เอาเงินไปใช้ประโยชน์แก่ตัวเอง

งานวิจัยนี้ทดลองในสามกลุ่มตัวอย่างนะครับ ทั้งในระดับประเทศ ระดับองค์กร และระดับนักศึกษามหาวิทยาลัย ปรากฎว่าผลออกมาตรงกันหมดเลยครับ ทุกกลุ่มตัวอย่างมีความสุขในการให้มากกว่าการรับจริง  ๆ

และในงานทดลองอีกชิ้นที่เกี่ยวข้องกันนั้นพบว่า ความสุขในการ ``มั่งมี" ของคนเรานั้นไม่ได้เกี่ยวข้องกับปริมาณที่มี แต่เกี่ยวข้องกับการเปรียบเทียบการ ``มั่งมี" นั้นกับคนอื่น

ผลการทดลองบอกว่า คนที่มีเงินร้อยล้านไม่ค่อยมีความสุขเมื่ออยู่กับคนที่มีเงินพันล้าน ส่วนคนที่มีเงินแสนที่อยู่ท่ามกลางเพื่อนฝูงที่มีเงินหมื่นมีความสุขในชีวิตมากกว่าคนที่มีเงินร้อยล้านนั้น

เรื่องนี้น่าสนใจอย่างยิ่ง เพราะเป็นการพิสูจน์ว่า หากอยากจะรวยแล้ว รวยยังไงก็ไม่มีวันได้ ``รวยจริงๆ" หรอก

และในอีกงานทดลองหนึ่งพบว่า ``การสูญเสียกระทบจิตใจรุนแรงมากกว่าการได้รับ" หมายความว่า หากคนหนึ่งได้เงินแสนจะมี "ความรุนแรง" ของความรู้สึกดีใจน้อยกว่า ``ความรุนแรง" ของความรู้สึกของคนนั้นหากต้องสูญเสียเงินหมื่นครับ

เรื่องเหล่านี้เป็นเรื่องทั่วไปในชีวิตประจำวันของเรา แต่คราวนี้นักวิจัยได้พิสูจน์ออกมาแล้วอย่างถูกต้องตามหลักวิทยาศาสตร์ ก็ยิ่งเป็นสิ่งย้ำเตือนให้เรารู้ว่าความสุขที่แท้จริงคืออะไรครับ

ท้ายสุดงานทดลองเหล่านี้พิสูจน์ว่า หากต้องการกระตุ้นสมองให้มีความสุขนั้น ใช้เงินไม่มากครับ ใช้เงินเพียง 5 ดอลล่าร์ต่อวันทำสิ่งที่มีประโยชน์ต่อคนอื่นเท่านั้นสมองก็ถูกกระตุ้นให้มีความสุขได้แล้วครับ

ผมลองมานึกปรับยอดเงิน 5 ดอลล่าร์มาเป็นเงินบาท ก็จะได้ประมาณ 30-40 บาทครับ เปรียบเทียบจากราคา BigMc ซึ่งราคาประมาณ 4.99 ดอลล่าร์ตอนสมัยผมยังอยู่ที่อเมริกา

ไม่ได้แปลงตามอัตราแลกเปลี่ยนนะครับ แต่แปลงตามการใช้ประโยชน์ของเงินจำนวนนั้นครับ คือ 5 ดอลล่าร์กินอาหารแบบราคาถูกในแคนาดา (สถานที่ทำวิจัยตามข่าว) ได้มื้อหนึ่ง 30-40 บาทก็กินอาหารในคุณภาพเดียวกันได้มื้อหนึ่งในประเทศไทยเช่นกันครับ

สรุปว่าถ้าเราใช้เงินเพื่อประโยชน์แก่คนอื่นประมาณวันละ 30-40 บาททุกวัน เราจะเป็นบุคคลที่มีความสุขทุกวันเลยครับ

คิดดูดี ๆ แล้ว ความสุขในชีวิตนี่หาง่ายจริง ๆ ทำไมคนบางคนถึงต้องดิ้นรนมากมายนักก็ไม่รู้ครับ

\newpage

\noindent\textbackslash postinfo\{ตัวอย่าง: พิสูจน์แล้ว การให้มีความสุขกว่าการรับ\}\{ผศ.ดร.ธวัชชัย ปิยะวัฒน์\}\{http://gotoknow.org/blog/averageline/175334\}\{4 เมษายน 2551\} \\

\noindent ข่าวใหญ่ประจำวันนี้ครับ เป็นข่าวใหญ่ในความรู้สึกผม แต่ไม่ได้เป็นข่าวใหญ่ที่สำนักข่าวไหน ๆ นำไปเป็นข่าวสำคัญครับ \\

\noindent ข่าวนั้นคือ ``นักวิทยาศาสตร์พิสูจน์แล้ว การให้มีความสุขกว่าการรับ เพราะการให้คือความสุขที่แท้จริง" \\

\noindent ผมอ่านข่าวนี้มาจาก \textbackslash link\{Giving = Happiness\}\{http://www.worldchanging.com/archives/007907.html\} เป็นข่าวที่น่ายินดีเป็นอย่างยิ่ง \\

\noindent นักวิจัยจาก University of British Columbia (UBC) ได้ค้นพบว่า ``การให้มีความสุขกว่าการรับ" ครับ โดยใช้การทดลองกับกลุ่มตัวอย่างโดยให้เงินแล้วทดสอบทางจิตวิทยาในภายหลัง พบว่ากลุ่มที่ให้นำเงินไปใช้ประโยชน์แก่คนอื่นเป็นกลุ่มที่สมองส่วนแสดงความสุขทำงานดีกว่ากลุ่มที่ให้เอาเงินไปใช้ประโยชน์แก่ตัวเอง \\

\noindent ...........................

\end{document}